%%%%%%%%%%%%%%%%
% $Id: place_and_route.tex,v 1.8 2009/08/26 12:25:11 jpc Exp $
% $Log: place_and_route.tex,v $
% Revision 1.8  2009/08/26 12:25:11  jpc
% Some more adjustments for LaTeX: no more here.sty, rule thickness put
% after begin{document} for fancyhdr.
%
% Revision 1.7  2009/08/26 11:41:55  jpc
% Replacing fancyheaders by fancyhdr for newer LaTeX's versions.
%
% Revision 1.6  2007/12/26 12:46:15  xtof
% ged rid of obsolete packages
%
% Revision 1.5  2004/10/16 12:51:56  fred
% Erasing the psfig include from the file, changed the font to 10 pt
% instead of 12 (sparing trees and not being payed by the thickness of
% my production) and changing font to charter since I got tired of
% Palatino, sorry Herman!
%
%%%%%%%%%%%%%%%%%%%%%%%%%%%%%%%%%%
\documentclass{article}
\usepackage[dvips]{graphics}
\usepackage[english]{babel}
\usepackage{setspace}
\usepackage{epsf}
\usepackage{fancybox}
\usepackage{fancyhdr}
\usepackage{float}
\usepackage{graphicx}
%\usepackage{here}
%\usepackage{isolatin1}
\usepackage{charter}
\usepackage{picinpar}
\usepackage{rotate}
\usepackage{subfigure}
\usepackage{sverb}
\usepackage{t1enc}
\usepackage{wrapfig}


\setlength{\topmargin}{0cm}
\setlength{\headheight}{1cm}
\setlength{\textheight}{23cm}
\setlength{\textwidth}{16cm}
\setlength{\oddsidemargin}{0cm}
\setlength{\evensidemargin}{0cm}
\setlength{\columnsep}{0.125in}
\setlength{\columnseprule}{0.5pt}
\setlength{\footskip}{1cm}
\setstretch{1.2}


%--------------------------------- styles
%--------------------------------
%
% Setting the width of the verbatim parts according to 80 tt chars
% Since it is tt, any char is fine
%
\newlength{\verbatimbox}
\settowidth{\verbatimbox}{\scriptsize\tt
xxxxxxxxxxxxxxxxxxxxxxxxxxxxxxxxxxxxxxxxxxxxxxxxxxxxxxxxxxxxxxxxxxxxxxxxxxxxxxxx
}

\newenvironment{sourcelisting}
  {\VerbatimEnvironment\par\noindent\scriptsize
   \begin{Sbox}\begin{minipage}{\verbatimbox}\begin{Verbatim}}%
  {\end{Verbatim}\end{minipage}\end{Sbox}

\setlength{\fboxsep}{3mm}\center\shadowbox{\TheSbox}\normalsize\par\noindent}

\newenvironment{commandline}
  {\VerbatimEnvironment\par\vspace*{2mm}\noindent\footnotesize
   \begin{Sbox}\begin{minipage}{.979\textwidth}\begin{Verbatim}}%
  {\end{Verbatim}\end{minipage}\end{Sbox}\setlength{\shadowsize}{2pt}%
  \shadowbox{\TheSbox}\normalsize\par\noindent}

%--------------------------------- page style --------------------------------
\pagestyle{fancy} \rhead{Place and route}
\lhead{PART 3} \rfoot{\thepage} \lfoot{ALLIANCE TUTORIAL} \cfoot{}
%
%\begin{figure}[H]\centering
% \includegraphics[width=8cm,height=8cm]{.eps}
% \caption{}
% \label{Fig:}
%\end{figure}
%---------------------------------- document ---------------------------------
\begin{document}
\setlength{\footrulewidth}{0.6pt}

\title{
               {\Huge ALLIANCE TUTORIAL \\}
    {\large
               Pierre \& Marie Curie University \\
                          2001 - 2004\\
    }
    \vspace{1cm}
    {\huge
                      PART 3\\
    		place and route
    }
}
\date{}
\author{
Frederic AK\hspace{2cm} Kai-shing LAM\\
Modified by LJ
}

\maketitle
\begin{figure}[H]\centering
  \includegraphics[height=10cm]{amd2901.epsi}
\end{figure}

\thispagestyle{empty}
\def\myfbox#1{\vspace*{3mm}\fbox{#1}\vspace{3mm}}
\newpage
{\bf Contents}\\
\\
{1} {\bf Introduction}
\\
{2 }{\bf Inverter and buffer drawing using GRAAL}

{2.1} Introduction

\hspace{0.5cm} {2.1.1} Technological environment

\hspace{0.5cm} {2.1.2} GRAAL

\hspace{0.5cm} {2.1.3} COUGAR

\hspace{0.5cm} {2.1.5} PROOF

{2.2} inverter Diagram

{2.3} Buffer diagram 

{2.4} sxlib gauge

{2.5} steps to follow

\hspace{0.5cm}  {2.5.1} Create an inverter

\hspace{0.5cm}  {2.5.2} Create a buffer
\\
{3} {\bf Place and Route}

{3.1} Amd2901 architecture

{3.2} Tools used

{3.3} Technological environment

{3.4} Beware of naming the files

{3.5} Data-path predefined placement

{3.6} heart Placement

{3.7} Route the heart

{3.8} pads placement
\\
{4} {\bf Annexes} 
 
\newpage
       {\huge
        PART 3 : }
        \vspace{1cm}
        {\huge
        Place and route
        }
 
All the files used in this part are located under \\
\texttt{/tutorial/place\_and\_route/src} directory.\\
This directory contents three subdirectories and one Makefile :

\begin{itemize}\itemsep=-.8ex

\item   Makefile
\item   inv
    \begin{itemize}\itemsep=-.8ex
    \item   Makefile
    \item   inv.vbe    : behavioral description 
    \item   inv\_x1.ap : inverter cell design using GRAAL
    \end{itemize}
\item   buffer
    \begin{itemize}\itemsep=-.8ex
    \item   Makefile
    \item   buffer.vbe : behavioral description
    \item   buf\_x2.ap : buffer cell design using GRAAL
    \end{itemize}
\item  amd2901 
    \begin{itemize}\itemsep=-.8ex
    \item   Makefile
    \item   amd2901\_ctl.vbe : behavioral description of control
    part
    \item   amd2901\_dpt.vbe : behavioral description of data-path
    \item   amd2901\_ctl.c : file .c of control part
    \item   amd2901\_dpt.c : file .c of data-path
    \item   amd2901\_core.c : file .c of heart
    \item   amd2901\_chip.c : file .c of the circuit with their
    pads
    \item   pattern.pat : tests file
    \end{itemize}
\end{itemize}


\newpage

\section{Introduction}
%---------------------
 The goal of this tutorial is to present some ALLIANCE tools :
\begin{itemize}\itemsep=-.4ex
\item   {\bf GRAAL} Graphic layout editor ;
\item   {\bf DRUC} Design rule checker ;
\item   {\bf COUGAR} Symbolic layout extractor ;
\item   {\bf PROOF} Formal proof between two behavioral descriptions ;
\item   {\bf OCP, OCR, NERO, RING} place and route tools .
\end{itemize}

The beginning of this tutorial will relate to the drawing under {
\bf GRAAL } of a inverter cell and a buffer.
The predefined cells concepts, model and
hierarchy will be introduced .\\
Then this tutorial contain the methodology used in Alliance to produce
the amd2901 physical layout that you conceived in Alliance Tutorial 
PART 2 "Synthesis" (All the documents used will be provided to you).
 
\newpage
%%%%%%%%%%%%%%%%%%%%%%%%%%%%%%%%%%%%%%%%%%%%%%%%%%%%%%%%%%%%%

\section{Inverter and buffer drawing under GRAAL}
%-----------------------------------------------

\subsection{Introduction}
%------------------------
    The library can be enriched by new cells with {\bf GRAAL} editor .\\
{ \bf GRAAL } is an editor of \/{\underline{symbolic }} {\it
layout} integrating the drawing rules checker {\bf DRUC} and also 
a net extractor.
 The first part here aims to draw an inverter cell inv\_x1 in the shape
of a predefined cell of sxlib complyiant with provided
drawing rules.

\subsubsection{Technological environment}
%--------------------
Some tools of Alliance use a particular technological
environment. It is indicated by the environment variable {\bf
RDS\_TECHNO\_NAME} which must be set to
{\bf/alliance/etc/cmos.rds}

\subsubsection{GRAAL}
%--------------------
    The {\it layout } editor \/handles six different objects types which we can create
    with the menu { \bf CREATE }:
\begin{itemize}\itemsep=-.4ex
\item   The ''instance'' (physical cells importation)
\item   The abutment boxes which define the cell limits 
\item   Segments: DiffN, DiffP, Poly, Alu1, Alu2... CAluX is used to specify
        a possible rectangle area for the connectors.
\item   VIAs or contacts: ContDiffN, ContDiffP, ContPoly and
    ViaMetal1/Metal2.
\item   Big VIAs
\item   Transistors: NMOS or PMOS
\end{itemize}

{\bf GRAAL} uses the environment variable {\bf
GRAAL\_TECHNO\_NAME}. It must be set to {\bf/alliance/etc/cmos.graal}.

Steps to follow to create a sxlib cell by respecting the sxlib gauge :
( cf 2.4 Sxlib gauge ) 
\begin{itemize}\itemsep=-.4ex
\item	place the supply Vdd and Vss using the menu CREATE->Segment
\item	place the VIAs using the menu CREATE->VIA
\item 	place the transistors PMOS and NMOS using the menu CREATE->Transistor
\item	place the NWell body using the menu CREATE->Segment
\item 	place the input/output connectors using the menu CREATE->VIA
\item	link the transistor P and the transistor N with the Poly segment using the menu CREATE->Segment
\item	supply each transistor by linking them with Ndiff and Pdiff segments and VIAs contacts
\item 	define the cell limit with an abutment box using the menu CREATE->Abutment Box
\end{itemize} 

\subsubsection{COUGAR}
%--------------------
The tool { \bf COUGAR } is able to extract the { \it netlist } from
a circuit to the format { \bf al } or { \bf spi } given a layout description
with the format { \bf ap }.
To extract a netlist at transistor level, use the following command :
\begin{commandline}
 > cougar -t file1 file2 
\end{commandline}

{ \bf COUGAR } uses the environment variables { \bf MBK\_IN\_PH }
and { \bf MBK\_OUT\_LO } according to the input and output formats.
For example to generate a SPICE netlist (with the format { \bf .spi })
starting from a layout description { \bf .ap } it is necessary to set
the following environment variables: \\

\begin{commandline}
 > MBK_IN_PH = ap 
 > export MBK_IN_PH 
 > MBK_OUT_LO = spi 
 > export MBK_OUT_LO
\end{commandline}

\begin{commandline}
 > cougar -t circuit circuit  
\end{commandline}

The resulting spice netlist can be then simulated using a SPICE simulator and a given
model card for a dedicated technology.

The schematic of the transistor neltlist can also be displayed using {\bf XSCH} :
\begin{commandline}
 > xsch -I spi -l circuit  
\end{commandline}

\subsection{inverter Diagram}
%---------------------------------

The theoretical inverter diagram is presented at the following
figure:

\begin{figure}[H]\centering
  \includegraphics[width=6cm]{inv_x1.eps}
  \caption{transistors diagram of a C-MOS inverter}
  \label{Fig:inv_x1}
\end{figure}

\subsection{Buffer diagram }
%---------------------------------

The theoretical buffer diagram is presented at the following
figure:

\begin{figure}[H]\centering
  \includegraphics[width=8cm]{buff_x1.eps}
  \caption{transistors diagram of a C-mos buffer}
  \label{Fig:buff_x1}
\end{figure}

It uses two inverter according to the hierarchy:

\begin{figure}[H]\centering
  \includegraphics[width=8cm]{hierarchie.eps}
  \caption{C-mos buffer hierarchy}
  \label{Fig:hier_x1}
`\end{figure}

\subsection{sxlib gauge}
%---------------------------------

\begin{itemize}\itemsep=-.4ex
\item The sxlib cells have whole 50 lambdas height and a multiple of 5 lambdas width.
\item The supply Vdd and Vss are carried out in Calu1; they have 6 lambdas width and are 
horizontally placed in top and bottom of the cell.
\item The transistors P are placed close to the Vdd while transistors N are placed close
        to the Vss.
\item Box N must have 24 lambdas height .
\item The special segments CAluX (CAlu1, Calu2, CAlu3...) form the cell interface (PORT\_MAP) 
and play the role of ''flat'' connectors. They must be placed on a 5x5 grid and can be anywhere in the cell.
\item The special segments TAlux (TAlu1, TAlu2...) are used to indicate the obstacles for the
        router. When you want to protect AluX segment, it is necessary to cover them
        or surround them by corresponding TAlux (same layer). TAluX are placed on a grid
        with 5 lambdas steps (figure \ref{Fig:gabarit2}).
\item The minimal width of CAlu1 is 2 lambda, plus 1 lambda for the extension (figure \ref{Fig:gabarit3}).
\item The boxes N and P must be polarized. { \bf It should be respectively connected to Vdd and Vss }.
\end{itemize}

You will find a summary of these constraints on the diagram 
\ref{Fig:gabarit}:

\begin{figure}[H]\centering
  \includegraphics[scale=0.8]{gabarit_sx.eps}
  \caption{a cell model of the sxlib library }
  \label{Fig:gabarit}
\end{figure}

\begin{figure}[H]\centering
  \includegraphics[scale=0.8]{gabarit2_sx.eps}
  \caption{Use the layer TAluX like protection}
  \label{Fig:gabarit2}
\end{figure}

\begin{figure}[H]\centering
  \includegraphics[scale=0.8]{gabarit3_sx.eps}
  \caption{Low size of CAlu1}
  \label{Fig:gabarit3}
\end{figure}

\subsection{steps to follow}
%---------------------------------

\subsubsection{Create an inverter}
%--------------------

\begin{itemize}\itemsep=-.4ex
\item   describe the cell inverter behavior in a file { \bf .vbe } .
\item   draw the inverter "stick-diagram" inv\_x1 whose transistors diagram
        is represented on the figure \ref{Fig:inv_x1}.

\begin{figure}[H]\centering
  \includegraphics[width=8cm]{stick.eps}
  \caption{stick diagram}
  \label{Fig:stick}
\end{figure}

\item   draw the cell under {\bf GRAAL} by respecting the
        gauge specified on the figure \ref{Fig:gabarit}.
\item   validate the symbolic drawing rules by launching {\bf DRUC} under {\bf GRAAL}.
\item   extract the { \it netlist } \/from the inverter to the format {\bf al} with {\bf COUGAR}.
\end{itemize}

\subsubsection{Create a buffer}
%--------------------

The buffer is produced under { \bf GRAAL } starting from the
instanciated of two inverters. The hierarchy thus created is
represented on the figure \ref{Fig:hier_x1}. The transistors
diagram is represented on the figure \ref{Fig:buff_x1}.

\begin{itemize}\itemsep=-.4ex
\item   describe the cell buffer behavior in a file { \bf .vbe }.
\item   draw the cell under {\bf GRAAL} by respecting the
        gauge specified on the figure \ref{Fig:gabarit}.
        You will use for that the instanciated function of { \bf GRAAL }.
        The cell with instanciate is of course the inverter, which you will connect (will routing) manually.
\item   validate the symbolic drawing rules by launching {\bf DRUC} under {\bf GRAAL}.
\item   extract the { \it netlist } \/from the buffer to the format {\bf al} with {\bf COUGAR}.
\end{itemize}

Do not forget that { \it man } pages exist...
We provide you the cells behaviour description inv.vbe and buffer.vbe;
and the cells inverter and buffer drawn under { \bf GRAAL }.

%%%%%%%%%%%%%%%%%%%%%%%%%%%%%%%%%%%%%%%%%%%%%%%%%%%%%%%%%%%%%

\section{Place and Route}
%-------------------

\subsection{Amd2901 architecture}
%---------------------------------

Am2901 breaks up into 2 blocks: the part controls which gathers
the logic `` glu '' (random logic) and the operative part (data-path).

\begin{figure}[H]\centering
  \includegraphics[scale=0.8]{bloc.eps}
  \caption{Amd decomposition in functional units}
  \label{Fig:decomposition}
\end{figure}


\begin{itemize}\itemsep=-.4ex
\item The data-path contains the regular parts of Amd2901, the registers
     and the arithmetic logic unit.
\item The control part contains irregular logic, 
    the instructions decoding and the `` flags '' computation.
\end{itemize}

The Hierarchy of descriptions is as follows:
\begin{figure}[H]\centering
  \includegraphics[scale=0.8]{hier.eps}
  \caption{Hierarchy used}
  \label{Fig:hierarchie}
\end{figure}

\subsection{Tools used}
%---------------------------------

You will use place and route tools { \bf ocp } and {\bf nero },
thus all tools for checking seen in the first part of this Tutorial .\\
{\bf ocp} is the placer, {\bf nero} allows routing over the cell.
The data-path and the control part will be placed and routed together and not separately. \\
You will use also {\bf lvx}, the netlists comparator. When the
system is too complex it is difficult to use {\bf proof}, the
formal comparator (calculations too long). A netlists comparison 
then is used. Test the two methods ({\bf proof} and {\bf
lvx}).

\subsection{Technological environment}
%---------------------------------

\begin{sourcelisting}
 > VH_MAXERR = 10
 > export VH_MAXERR
 > MBK_WORK_LIB = .
 > export MBK_WORK_LIB
 > MBK_CATA_LIB = $ALLIANCE_TOP/cells/sxlib
 > export MBK_CATA_LIB
 > MBK_CATA_LIB = $MBK_CATA_LIB:$ALLIANCE_TOP/cells/dp_sxlib
 > export MBK_CATA_LIB
 > MBK_CATA_LIB = $MBK_CATA_LIB:$ALLIANCE_TOP/cells/padlib
 > export MBK_CATA_LIB
 > MBK_CATA_LIB $MBK_CATA_LIB:.
 > export MBK_CATA_LIB
 > MBK_CATAL_NAME = CATAL
 > export MBK_CATAL_NAME
 > MBK_IN_LO = vst
 > export MBK_IN_LO
 > MBK_OUT_LO = vst
 > export MBK_OUT_LO
 > MBK_IN_PH = ap
 > export MBK_IN_PH
 > MBK_OUT_PH = ap
 > export MBK_OUT_PH
\end{sourcelisting}

\subsection{Beware of file naming}
%---------------------------------

Generally, the file describing a netlist must have the same
name as the one describing its physical layout 
(but of course the file extention is not the same).
The file amd2901\_dpt.vst (LOFIG) must correspond to the file
amd2901\_dpt.ap (PHFIG). The same applies to the file
amd2901\_core. Be carefull not to overwrite a file by mistake !

\subsection{Data-path predefined placement}
%---------------------------------

For the moment, your file amd2901\_dpt.c describes only
the netlist.
eg you have a C file that contains the following lines: \\

\noindent GENLIB\_DEF\_LOFIG()\\
\noindent ...\\
\noindent GENLIB\_SAVE\_LOFIG()\\

This permits to generate a structural description in a { \bf
VST } file. At the same time, { \bf genlib } will generate 
physical descriptions of each column in { \bf AP } files.
It is up to you to place these columns explicitly. \\
Edit again the file amd2901\_dpt.c and include the lines :\\

\noindent GENLIB\_DEF\_PHFIG()\\
\noindent /* add here you placement directives !! */ \\
\noindent GENLIB\_SAVE\_PHFIG()\\

For this placement task, you have the following {\bf GENLIB} functions : 

\begin{itemize}\itemsep=-.4ex
\item GENLIB\_PLACE()
\item GENLIB\_PLACE\_RIGHT()
\item GENLIB\_PLACE\_TOP()
\item GENLIB\_PLACE\_LEFT()
\item GENLIB\_PLACE\_BOTTOM()
\item GENLIB\_PLACE\_ON()
\item GENLIB\_DEF\_AB()
\item ...
\end{itemize}

Use {\bf GENLIB} manual. The placement of the  data-path columns
should not be done randomly. The routing feasibility and the quality 
of the resulting layout depends on it !\\

Use genlib to generate all:
\begin{commandline}
 >genlib  amd2901_dpt 
\end{commandline}

The figure \ref{Fig:preplacement} summarizes the followed process:

\begin{figure}[H]\centering
  \includegraphics[scale=0.8]{preplacement.eps}
  \caption{predefined placement}
  \label{Fig:preplacement}
\end{figure}

\begin{figure}[H]\centering
  \includegraphics[scale=0.8]{colonnes.eps}
  \caption{predefined Columns before placement of the part controls }
  \label{Fig:colonnes}
\end{figure}

Do not forget to include a abutment box!

\subsection{heart Placement}
%---------------------------------

In the same manner, edit agin the file amd2901\_core.c and insert
 data-path explicitly. You should not place the part controls. 
This one exists only in the form of a structural description.
It is the placer { \bf ocp } that will undertake some 
(during the placement of the heart { \bf ocp } detects which are the
cells not placed and supplements the placement). 
Nevertheless you should reserve enough space for the cells placement 
{ \bf to the top } of the data-path.

Include the lines:\\
\noindent GENLIB\_DEF\_PHFIG()\\
\noindent /* add here placement directives for your data-path */\\
\noindent GENLIB\_SAVE\_PHFIG()\\

Space necessary to the placer to place the cells of the control part
will be determined by successive approximations. You will have to
adjust dimensions of the heart abutment box 
(GENLIB\_DEF\_AB()).
Use the command:

\begin{commandline}
 > genlib  amd2901_core 
\end{commandline}

and
\begin{commandline}
 > ocp -partial amd2901_core -ioc amd2901_core amd2901_core amd2901_core_p 
\end{commandline}

The option {\bf -- partial} indicates that you give a partial
placement of the data-path. 
The option { \bf -- ioc } permits to specify a placement for external
connectors described in a .ioc file. 
This file, amd2901\_core.ioc is provided to you (Modify it according 
to your predefined placement.
The connectors must be in the north and in the south of your circuit).

The third argument is the netlist heart filename, the fourth is the 
name of the { \bf .ap } resulting file.

The figure \ref{Fig:placement} summarize the followed process:
\begin{figure}[H]\centering
  \includegraphics[scale=0.6]{placement.eps}
  \caption{Placement}
  \label{Fig:placement}
\end{figure}

\subsection{Route the heart}
%---------------------------------

Routing the heart by using { \bf NERO } in the following way:

\begin{commandline}
 > nero -v -3 -p amd2901_core_p amd2901_core amd2901_core 
\end{commandline}

%The option { \bf -- place } indicates that you transmit a placement, that of the heart.
%The third argument is the netlist heart, the fourth is the file { \bf AP } result. \\

%{\bf NOTA}:
%\begin{itemize}\itemsep=-.4ex
%\item   Variable MBK\_CATA\_LIB should contain only once the access paths to the libraries.
%\item   The file of error generated by { \bf ocr } is {\bf .log}.
%{ \bf THIS FILE MUST BE IMPERATIVELY CONSULTS } before beginning
%the following stages. Indeed, it is thanks to this file that the
%router informs you if it made a success of with routing all the
%signals or not...
%\end{itemize}

\subsection{pads placement}
%---------------------------------

The core of the AMD2001 is completed. 
We focus now on the chip with pads description, placement and routing.
Those pads allow the connection of the inputs/outputs of the core with
the external nets of the chip. 

The tool {\bf ring} instanciates pads that has been specified
in a {\bf vst} netlist, place them using a file { \bf .rin }
that specified a relative placement of those pads.
It then routes those pads with the core according to the input
netlist.

This syntax of the {\bf .rin} file:
\begin{sourcelisting}
 > east ( pi1 pi0 )
 > west ( pck pi4 )
 > north ( pvdd pvss )
 > south ( pvdde pvsse )
\end{sourcelisting}

Where pi1, pi0... are the name of instance pads.
Name it `` amd2902\_chip.rin '' and apply the command \\

\begin{commandline}
 > ring amd2901_chip amd2901_chip 
\end{commandline}

We will validate the work of {\bf ring} with the tools { \bf druc
}, { \bf lynx } and { \bf lvx }.\\

Validate the physical design rules:
\begin{commandline}
 > druc amd2901_chip
\end{commandline}
Extract the netlist up to leave cells:
\begin{commandline}
 > MBK_OUT_LO = al
 > export MBK_OUT_LO
\end{commandline}

\begin{commandline}
 > cougar -f amd2901_chip
\end{commandline}

Compare two netlists : 
\begin{commandline}
 > lvx vst al amd2901_chip amd2901_chip -f
\end{commandline}

\begin{commandline}
 > MBK_OUT_LO = vst
 > export MBK_OUT_LO
\end{commandline}

Simulated the extracted netlist with { \bf asimut }.
Pay attention to the file { \bf CATAL }!\\
To know the number of transistors, we carry out an extraction of
the circuit on the level transistor: \\

\begin{commandline}
> cougar -t amd2901_chip amd2901_chip 
\end{commandline}
\\

If you want to see the amd2901 control part :
\begin{commandline}
> make view_ctl_logic
\end{commandline}
 
If you want to see the data-path physical layout:
\begin{commandline}
> make view_dpt_physic
\end{commandline}

note: you can see in red the critical path.

If you want to see the chip physical layout:
\begin{commandline}
> make view_chip_physic
\end{commandline}

If you want to see the different propagation times:
\begin{commandline}
> make view_chip_simulation
\end{commandline}
     
%%%%%%%%%%%%%%%%%%%%%%%%%%%%%%%%%%%%%%%%%%%%%%%%%%%%%%%%%%%%%

%\newpage
%\section{Conclusion}
%-------------------
%    This Tutorial enabled you to pass by the majority of the stages necessary to
%    the design and the validation of a circuit carried out in precaracterized cells. \\


\newpage

\section{Annexes}
%-------------------

\begin{figure}[H]\centering
  \includegraphics[scale=0.8]{dpt-all-1.eps}
  \caption{data-path general view }
  \label{Fig:dpt}
\end{figure}

%%%%%%%%%%%%%%%%%%%%%%%%%%%%%%%%%%%%%%%%%%%%%%%%%%%%%%%%%%%%%
%%%%%\cleardoublepage
\newpage

%\vfill

\newpage

%\vfill

\end{document}
